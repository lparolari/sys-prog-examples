%%%%%%%%%%%%%%%%%%%%%%%%%%%%%%%%%%%%%%%%%%%%%%%%%%%
%% LaTeX book template                           %%
%% Author:  Amber Jain (http://amberj.devio.us/) %%
%% License: ISC license                          %%
%%%%%%%%%%%%%%%%%%%%%%%%%%%%%%%%%%%%%%%%%%%%%%%%%%%

%%%%%%%%%%%%%%%%%%%%%%%%%%%%%%%%%%%%%%%%%%%%%%%%%%%%%%%
% Sample table                                        %
% Source: www1.maths.leeds.ac.uk/latex/TableHelp1.pdf %
%%%%%%%%%%%%%%%%%%%%%%%%%%%%%%%%%%%%%%%%%%%%%%%%%%%%%%%
%\begin{table}[ht]
%\caption{Sample table} % title of Table
%\centering % used for centering table
%\begin{tabular}{c c c c}
% centered columns (4 columns)
%\hline\hline %inserts double horizontal lines
%S. No. & Column\#1 & Column\#2 & Column\#3 \\ [0.5ex]
%% inserts table
%%heading
%\hline % inserts single horizontal line
%1 & 50 & 837 & 970 \\
%2 & 47 & 877 & 230 \\
%3 & 31 & 25 & 415 \\
%4 & 35 & 144 & 2356 \\
%5 & 45 & 300 & 556 \\ [1ex] % [1ex] adds vertical space
%\hline %inserts single line
%\end{tabular}
%\label{table:nonlin} % is used to refer this table in the text
%\end{table}

\documentclass[a4paper,11pt,oneside]{book}
\usepackage[T1]{fontenc}
\usepackage[utf8]{inputenc}
\usepackage{lmodern}

\usepackage{hyperref} % http://en.wikibooks.org/wiki/LaTeX/Hyperlinks
\usepackage{graphicx}
\usepackage[english]{babel}

\usepackage{answers}
\usepackage{setspace}
\usepackage{enumitem}
\usepackage{multicol}
\usepackage{mathrsfs}
\usepackage[margin=1in]{geometry} 
\usepackage{amsmath,amsthm,amssymb}
\usepackage[useregional]{datetime2}
\usepackage{listings}
\usepackage{verbatim}
\usepackage{fancyref}

% %%%%%%%%%%%%%%%%%%%%%%%%%%%%%%%%%%%%%%%%%%%%%%%%%%%%%%%
%
% code and colors packages
%
% https://stackoverflow.com/questions/3175105/writing-code-in-latex-document?utm_medium=organic&utm_source=google_rich_qa&utm_campaign=google_rich_qa
%
% %%%%%%%%%%%%%%%%%%%%%%%%%%%%%%%%%%%%%%%%%%%%%%%%%%%%%%%

\usepackage{listings}	% code
\usepackage{xcolor}		% color

\definecolor{dkgreen}{rgb}{0,0.6,0}
\definecolor{gray}{rgb}{0.5,0.5,0.5}
\definecolor{mauve}{rgb}{0.58,0,0.82}

\lstset{
	%frame=tb,
	language=c++,
	aboveskip=3mm,
	belowskip=3mm,
	showstringspaces=false,
	columns=flexible,
	basicstyle={\small\ttfamily},
	numbers=none,
	%numberstyle=\tiny\color{gray},
	%keywordstyle=\color{blue},
	%commentstyle=\color{dkgreen},
	%stringstyle=\color{mauve},
	breaklines=true,
	breakatwhitespace=true,
	tabsize=3
}

\def\codeinline{
	\lstinline[
		basicstyle=\ttfamily,
		keywordstyle={}
	]
}

%%%%%%%%%%%%%%%%%%%%%%%%%%%%%%%%%%%%%%%%%%%%%%%%%%%%%%%%%%%%%%%%%%%%%%%%%%%%%%%%
% 'dedication' environment: To add a dedication paragraph at the start of book %
% Source: http://www.tug.org/pipermail/texhax/2010-June/015184.html            %
%%%%%%%%%%%%%%%%%%%%%%%%%%%%%%%%%%%%%%%%%%%%%%%%%%%%%%%%%%%%%%%%%%%%%%%%%%%%%%%%
\newenvironment{dedication}
{
   \cleardoublepage
   \thispagestyle{empty}
   \vspace*{\stretch{1}}
   \hfill\begin{minipage}[t]{0.66\textwidth}
   \raggedright
}
{
   \end{minipage}
   \vspace*{\stretch{3}}
   \clearpage
}

%%%%%%%%%%%%%%%%%%%%%%%%%%%%%%%%%%%%%%%%%%%%%%%%
% Chapter quote at the start of chapter        %
% Source: http://tex.stackexchange.com/a/53380 %
%%%%%%%%%%%%%%%%%%%%%%%%%%%%%%%%%%%%%%%%%%%%%%%%
\makeatletter
\renewcommand{\@chapapp}{}% Not necessary...
\newenvironment{chapquote}[2][2em]
  {\setlength{\@tempdima}{#1}%
   \def\chapquote@author{#2}%
   \parshape 1 \@tempdima \dimexpr\textwidth-2\@tempdima\relax%
   \itshape}
  {\par\normalfont\hfill--\ \chapquote@author\hspace*{\@tempdima}\par\bigskip}
\makeatother

%%%%%%%%%%%%%%%%%%%%%%%%%%%%%%%%%%%%%%%%%%%%%%%%%%%
% First page of book which contains 'stuff' like: %
%  - Book title, subtitle                         %
%  - Book author name                             %
%%%%%%%%%%%%%%%%%%%%%%%%%%%%%%%%%%%%%%%%%%%%%%%%%%%

% Book's title and subtitle
\title{\Huge \textbf{Operative Systems}  \footnote{Operative Systems course at UNIPR.} \\ \huge Workbook \footnote{Report and solved exercises from slides seen at lessons.}}
% Author
\author{\textsc{Luca Parolari}\thanks{\href{mailto:luca.parolari@studennti.unipr.it}{luca.parolari@studennti.unipr.it}}}


\begin{document}

\frontmatter
\maketitle

%%%%%%%%%%%%%%%%%%%%%%%%%%%%%%%%%%%%%%%%%%%%%%%%%%%%%%%%%%%%%%%
% Add a dedication paragraph to dedicate your book to someone %
%%%%%%%%%%%%%%%%%%%%%%%%%%%%%%%%%%%%%%%%%%%%%%%%%%%%%%%%%%%%%%%
%\begin{dedication}
%Dedicated to Calvin and Hobbes.
%\end{dedication}

%%%%%%%%%%%%%%%%%%%%%%%%%%%%%%%%%%%%%%%%%%%%%%%%%%%%%%%%%%%%%%%%%%%%%%%%
% Auto-generated table of contents, list of figures and list of tables %
%%%%%%%%%%%%%%%%%%%%%%%%%%%%%%%%%%%%%%%%%%%%%%%%%%%%%%%%%%%%%%%%%%%%%%%%
\tableofcontents
%\listoffigures
%\listoftables

\mainmatter



% ----------------------------------------------------------------------------
%%%%%%%%%%%%%%%%%%%%%%%%%%%%%%%%%%%%%%%%%%%%%%%%
%											   %
% 				 	Preface					   %
%											   %
%%%%%%%%%%%%%%%%%%%%%%%%%%%%%%%%%%%%%%%%%%%%%%%%

\chapter*{Preface}
First of all, this paper will be a \textbf{workbook} where i do exercises from slides, but the second goal is to write a little "manual" for basic commands and unix details on which i can study.

\section*{Why in english?}
In a world where english is the first international language it's a good this to do some practise with a king of formal document, so i decided to write this workbook in english.
I apologize in advance if my english is not correct or if it sounds bad, but it's my first time!

\section*{Exercises}
Exercises are taken from a slide which can be found on \textit{link}.

\section*{Structure of document}
Each unit will focus on a particular exercise or group of exercises done at lesson but all section will be integrated with theory hints. \\

\noindent The theory concept are taken from my notes, from the slides provided on \href{http://elly.smfi.unipr.it/2017/pluginfile.php/138/course/section/805/Lezione\%201.pdf}{elly} and, if necessary, from the internet (sources will be listed in sitography).


\section*{Useful links}
\begin{itemize}
  \item \href{}{This document}.
  \item \href{}{This document source}.
  \item \href{http://elly.smfi.unipr.it/2017/course/view.php?id=9}{Operative System course on Elly at UNIPR}.
  
  \item Miscellaneous material (e.g. suggested readings etc).
\end{itemize}



% ----------------------------------------------------------------------------
%%%%%%%%%%%%%%%%%%%%%%%%%%%%%%%%%%%%%%%%%%%%%%%%
%											   %
% 				 Introduction				   %
%											   %
%%%%%%%%%%%%%%%%%%%%%%%%%%%%%%%%%%%%%%%%%%%%%%%%

\chapter{Introduction}
\begin{chapquote}{10/04/2018}
	Laboratory: Lesson 1
\end{chapquote}

\noindent This chapter will talk about fundamental pratical concepts of Linux/Unix OS.
In particular, i will talk about linux shell's types and characteristics. Then i will proceed with popular and useful commands like \codeinline{ls}, \codeinline{cd}, \codeinline{man}, \codeinline{ps}, and many others.
After that i will show an overview of Unix file structure and organization and little description for file permissions.


% *** Shell *** 	
\section{Shell}
The shell is a characters interface. 
The user can interact with it by executing commands and the shell responds to the user with some messages.

There are many types of shell:
\begin{itemize}
	\item \textbf{sh}: Bourne shell.
	\item \textbf{bash}: Bourne again shell.
	\item \textbf{csh}: C shell.
	\item \textbf{tcsh}: Teach C shell.
	\item \textbf{ksh}: Korn shell.
\end{itemize}

\noindent All the shells provides a programming language, that is very more powerful compared to visual interaction.


% *** Commands *** 
\section{Commands}
A command is something that the machine must execute.
Commands has this synopsis: \codeinline{command [options] <arguments>}.

Here there are some commands:
\begin{itemize}
\item \codeinline{date}, shows current date.
\begin{lstlisting}
foo@vtest:~$ date
mar 10 apr 2018, 21.12.38, CEST
\end{lstlisting}
	
\item \codeinline{who}, shows users connected to the system.
\begin{lstlisting}
foo@vtest:~$ who
foo  tty2  2018-04-10 16:47 (:1)
\end{lstlisting}
	
\item \codeinline{uname -a}, shows system informations.
\begin{lstlisting}
foo@vtest:~$ uname -a
Linux pc-bcvdev 4.13.0-38-generic #43-Ubuntu SMP Wed Mar 14 15:20:44 UTC 2018 x86_64 x86_64 x86_64 GNU/Linux
\end{lstlisting}
	
\item \codeinline{ps}, shows user's processes
\begin{lstlisting}
foo@vtest:~$ ps
PID TTY          TIME CMD
5154 pts/0    00:00:00 bash
12646 pts/0    00:00:00 ps

\end{lstlisting}
	
\item \codeinline{ps ef}, shows all processes
\begin{lstlisting}
foo@vtest:~$ ps ef
PID TTY      STAT   TIME COMMAND
5154 pts/0    Ss     0:00 bash [...]
12688 pts/0   R+     0:00  \_ ps ef
1678 tty2     Ssl+   0:00 /usr/lib/gdm3/gdm-x-session
1772 tty2     Sl+    0:00  \_ /usr/lib/gnome-session/gnome-session-binary
1907 tty2     Rl+   19:58      \_ /usr/bin/gnome-shell
1944 tty2     Sl     1:22      |   \_ ibus-daemon
1950 tty2     Sl     0:00      |   |   \_ /usr/lib/ibus/ibus-dconf
2221 tty2     Sl     0:25      |   |   \_ /usr/lib/ibus/ibus-.. 
6524 tty2     Sl+    4:59      |   \_ gummi 
10948 tty2    Sl+    0:04      |   \_ /usr/bin/emacs25

[MORE]
\end{lstlisting}

This command shows lots of more informations and lots of more processes but these are omitted for clearity.
	
\end{itemize}


% *** Files ***
\section{Files}

% Folder structure
\subsection{Folder structure}

Unix usually has standard and repetitive files structure which includes:

\begin{table}[ht]
\caption{Folder structure} % title of Table
\centering % used for centering table
\begin{tabular}{c c} % centered columns (2 columns)
\hline\hline %inserts double horizontal lines
Name & Description \\ [0.5ex]
% inserts table
%heading
\hline % inserts single horizontal line
/ & Root directory \\
bin & User commands \\
sbin & Admin commands \\
dev & I/O devices \\
etc & Configuration files \\
lib & Software library \\
var & Variable dimension files (ex. logs, mailbox..) \\
usr & Programs and apps \\
home & Personal directories for users \\
proc & Dynamic system informations \\
 [1ex] % [1ex] adds vertical space
\hline %inserts single line
\end{tabular}
\label{table:folderstructure} % is used to refer this table in the text
\end{table}

% File permissions
\subsection{File permissions}
Linux is a multiuser system.
There are four user categories: \textbf{root}, which has all permissions on all files and \textbf{owner} (u), \textbf{group} (g), \textbf{world} (o) 
which are regulated by permission rules.

\noindent Every file has an information made of \emph{10 bits} where: 
\begin{itemize}
\item \emph{the first bit} represents the file type (\emph{file} or \emph{directory});
\item bits \emph{from 2 to 4} represent owner permissions;
\item bits \emph{from 5 to 7} represent group permissions;
\item bits \emph{from 8 to 10} represent world permissions;
\end{itemize}

\noindent where in this group the first bit is for \emph{read}, the second for \emph{write} and the third for \emph{execute} permission.


% *** Exerices ***
\section{Exercises}

1. What is your home directory?

My home directory is located in \codeinline{ /home/foo/ }.
\\

\noindent 2. Show home directory file ordered by modification date.

\begin{lstlisting}
foo@vtest:~$ ls --sort=t -a -l
totale 177564
drwxr-xr-x  13 foo    foo        4096 apr 10 18:44 Scaricati
drwxr-xr-x  38 foo    foo        4096 apr 10 18:34 .config
drwxr-xr-x  52 foo    foo        4096 apr 10 16:47 .
-rw-------   1 foo    foo       47206 apr 10 16:47 .ICEauthority
-rw-------   1 foo    foo       43169 apr 10 11:29 .bash_history
lrwxrwxrwx   1 foo    foo          23 apr 10 10:54 omega -> /media/foo/Omega/
drwxrwxr-x   5 foo    foo        4096 apr  9 22:14 VirtualBox VMs
drwx------  38 foo    foo        4096 apr  9 18:34 .cache
drwx------   2 foo    foo        4096 apr  8 19:02 .gconf
drwxr-xr-x 985 foo    foo       36864 apr  6 15:30 .npm
-rw-------   1 foo    foo           0 apr  6 15:26 .node_repl_history
-rw-------   1 foo    foo         537 apr  4 20:13 .xdvirc
drwxr-xr-x  24 foo    foo        4096 mar 31 15:29 .gimp-2.8
drwxr-xr-x   5 foo    foo        4096 mar 30 18:04 .android
drwxr-xr-x   3 foo    foo        4096 mar 30 10:23 Scrivania
drwxr-xr-x   4 foo    foo        4096 mar 28 12:50 .wine
drwxr-xr-x   2 foo    foo        4096 mar 28 12:50 .swipl-dir-history
drwxr-xr-x   2 foo    foo        4096 mar 28 08:37 Video
-rw-r--r--   1 foo    foo         502 mar 27 16:16 .emacs
drwx------   5 foo    foo        4096 mar 26 22:26 .emacs.d
-rw-------   1 foo    foo          34 mar 26 21:18 .lesshst
-rw-------   1 foo    foo          71 mar 24 14:23 .sqlite_history
drwxr-xr-x   3 foo    foo        4096 mar 22 16:42 .texlive2017
-rw-r--r--   1 foo    foo         664 mar 22 16:32 texput.log
drwxr-xr-x  10 foo    foo        4096 mar 20 21:36 Documenti
drwxr-xr-x   2 foo    foo        4096 mar  8 15:35 Immagini
drwx------   2 foo    foo        4096 mar  8 15:05 .ssh
-rw-rw-r--   1 foo    foo          85 mar  5 12:04 .gitconfig
-rw-r--r--   1 foo    foo           0 mar  1 17:38 .odbc.ini
-rw-r-----   1 foo    foo           0 feb 28 17:42 .gksu.lock
drwxrwxr-x   3 foo    foo        4096 feb 27 18:55 .m2
drwxrwxr-x   3 foo    foo        4096 feb 27 17:39 shared
-rw-------   1 foo    foo        1024 feb 22 17:22 .rnd
-rw-rw-r--   1 foo    foo        3360 feb 14 10:37 .pgadmin3
-rw-rw-r--   1 foo    foo         204 feb 14 09:12 .pgadmin...
-rw-rw-r--   1 foo    foo         834 feb 14 09:12 pgadmin.log
-rw-------   1 foo    foo           0 feb 14 09:02 .pgpass
-rw-r--r--   1 foo    foo          73 feb 14 08:52 .selected_editor
-rw-------   1 foo    foo           0 feb 13 22:40 .psql_history
drwx------   2 foo    foo        4096 feb  8 14:39 .remmina
drwxr-xr-x   2 foo    foo        4096 feb  8 14:38 .vnc
drwx------   3 foo    foo        4096 feb  7 21:37 .gnupg
-rw-r--r--   1 root   root    2238078 feb  7 21:08 .-02.cap
-rw-r--r--   1 root   root   10908742 feb  7 21:08 .-01.cap
-rw-r--r--   1 root   root        597 feb  7 21:08 .-01.kismet.csv
-rw-r--r--   1 root   root       3792 feb  7 21:08 .-01.kismet.netxml
-rw-r--r--   1 root   root       4870 feb  7 21:08 .-02.kismet.netxml
-rw-r--r--   1 root   root        585 feb  7 21:08 .-01.csv
-rw-r--r--   1 root   root        686 feb  7 21:08 .-02.csv
-rw-r--r--   1 root   root        604 feb  7 21:08 .-02.kismet.csv
-rw-rw-r--   1 foo    foo        2423 feb  2 14:55 .overgrive.log
-rw-rw-r--   1 foo    foo           5 feb  2 14:54 .overgrive.lock
-rw-------   1 foo    foo          13 gen 30 16:12 .mysql_history
-rw-------   1 foo    foo        8192 gen 26 19:01 lp2s0
-rw-------   1 foo    foo        8192 gen 26 19:01 ls20
drwx------   3 foo    foo        4096 gen 26 08:12 .putty
drwx------   3 foo    foo        4096 gen 25 16:34 .gnome
-rw-rw-r--   1 foo    foo      155054 gen 25 13:31 java...
drwxrwxr-x   3 foo    foo        4096 gen 22 23:06 .tixati
drwxr-xr-x   2 foo    foo        4096 gen 22 22:23 Desktop
drwxr-xr-x   3 foo    foo        4096 gen 22 22:23 snap
drwxrwxr-x   2 foo    foo        4096 gen 22 21:41 MEGAsync
drwx------   3 foo    foo        4096 gen 22 13:58 .thumbnails
drwx------   4 foo    foo        4096 gen 22 13:57 .nv
drwx------   2 root   root       4096 gen 22 13:27 .gvfs
drwxrwxr-x   3 foo    foo        4096 gen 21 12:19 Sviluppo
-rw-r--r--   1 foo    foo         401 gen 21 12:13 id_rsa_20170121.pub
-rw-------   1 foo    foo        1766 gen 21 12:13 id_rsa_20170121
drwxr-xr-x   2 foo    foo        4096 gen 20 20:23 .swt
drwxr-xr-x   4 foo    foo        4096 gen 20 19:51 .smartgit
drwx------   4 foo    foo        4096 gen 20 18:18 .thunderbird
drwx------   3 foo    foo        4096 gen 20 18:14 .mysql
drwx------   3 root   root       4096 gen 20 18:04 .dbus
drwxr-xr-x   4 foo    foo        4096 gen 20 18:02 .IntelliJIdea2017.2
drwxr-xr-x   4 foo    foo        4096 gen 20 17:53 .PhpStorm2017.2
drwxr-xr-x   7 foo    foo        4096 gen 20 17:48 .gradle
-rw-------   1 foo    foo   168026112 gen 20 17:20 core
drwxrwxr-x   3 foo    foo        4096 gen 20 17:20 .vscode
drwx------   3 foo    foo        4096 gen 20 17:20 .pki
drwxr-xr-x   3 foo    foo        4096 gen 20 17:19 Android
drwxr-xr-x   4 foo    foo        4096 gen 20 17:18 .java
drwxr-xr-x   4 foo    foo        4096 gen 20 17:18 .AndroidStudio3.0
drwx------   5 foo    foo        4096 gen 20 15:30 .mozilla
-rw-r--r--   1 foo    foo           0 gen 20 15:28 .s ...
drwxr-xr-x   2 foo    foo        4096 gen 20 14:28 Modelli
drwxr-xr-x   2 foo    foo        4096 gen 20 14:28 Musica
drwxr-xr-x   2 foo    foo        4096 gen 20 14:28 Pubblici
drwxr-xr-x   3 foo    foo        4096 gen 20 14:28 .local
-rw-r--r--   1 foo    foo         220 gen 20 13:21 .bash_logout
-rw-r--r--   1 foo    foo        3771 gen 20 13:21 .bashrc
-rw-r--r--   1 foo    foo        8980 gen 20 13:21 examples.desktop
-rw-r--r--   1 foo    foo         675 gen 20 13:21 .profile
drwxr-xr-x   3 root   root       4096 gen 20 13:21 ..

\end{lstlisting}


\noindent 3. What is the difference between \codeinline{cat}, \codeinline{more} and \codeinline{tail}?

\codeinline{Cat} is a program used to concatenate files and print them on standard output. 

\codeinline{Tail} instead print  the  last  10  lines of each file to standard output.

And \codeinline{more} is a filter for paging through text one screenful at a time. \\

\noindent 4. Find a way to show home subdirectories recursively.

There are more than one way.

\begin{itemize}
\item \codeinline{ls -R} shows directories recursively;
\item \codeinline{ls -R | grep ":$" | sed -e 's/:$//' -e 's/[^-][^\/]*\//--/g' -e 's/^/   /' -e 's/-/|/'} found on google, manipulates the output of command described before and output something like this:
\begin{lstlisting}
   |-Android
   |---Sdk
   |-----build-tools
   |-------21.1.2
   |---------lib
   |---------renderscript
   |-----------clang-include
   |-----------include
   |-----------lib
   |-------------bc
   |---------------armeabi-v7a
   |---------------mips
   |---------------x86
   |-------------intermediates
   |---------------armeabi-v7a
   |---------------mips
   |---------------x86
   |-------------packaged
   |---------------armeabi-v7a
   |---------------mips
   |---------------x86
   |-------26.0.2
   |---------lib
   |---------lib64
   |---------renderscript
   |-----------clang-include
   |-----------include
   |-----------lib

[MORE]

\end{lstlisting}

\item \codeinline{tree} (installed with \codeinline {sudo apt-get install tree}) shows something like this:
\begin{lstlisting}
+-- Android
|   +-- Sdk
|       +-- build-tools
|       |   +-- 21.1.2
|       |   |   +-- aapt
|       |   |   +-- aidl
|       |   |   +-- arm-linux-androideabi-ld
|       |   |   +-- bcc_compat
|       |   |   +-- dexdump
|       |   |   +-- dx
|       |   |   +-- i686-linux-android-ld
|       |   |   +-- jack.jar
|       |   |   +-- jill.jar
|       |   |   +-- lib
|       |   |   |   +-- dx.jar
|       |   |   |   +-- shrinkedAndroid.jar
|       |   |   +-- libbcc.so
|       |   |   +-- libbcinfo.so
|       |   |   +-- libclang.so
|       |   |   +-- libc++.so
|       |   |   +-- libLLVM.so
|       |   |   +-- llvm-rs-cc
|       |   |   +-- mainDexClasses
|       |   |   +-- mainDexClasses.rules
|       |   |   +-- mipsel-linux-android-ld
|       |   |   +-- NOTICE.txt
|       |   |   +-- package.xml
|       |   |   +-- renderscript

[MORE]
\end{lstlisting}

\end{itemize}

\noindent 5. What following commands do?
\begin{lstlisting}
1: > cd
2: > mkdir d1
3: > chmod 444 d1
4: > cd d1
\end{lstlisting}

At \emph{line 1} the current dir is setted to home (tilde); at \emph{line 2} a new directory called \codeinline{d1} is created and then, at \emph{line 3} it's permissions are setted. The permissions say that all can read but nobody can write or execute the directory.
At \emph{line 4} is returned an error which says that we don't have permission to "run" (i.e. access) the directory.



% ----------------------------------------------------------------------------
%%%%%%%%%%%%%%%%%%%%%%%%%%%%%%%%%%%%%%%%%%%%%%%%
%											   %
% 				    Bash	 				   %
%											   %
%%%%%%%%%%%%%%%%%%%%%%%%%%%%%%%%%%%%%%%%%%%%%%%%

\chapter{Bash}

\begin{chapquote}{17/04/2018}
	Laboratory: Lesson 2
\end{chapquote}

\section{Exercises}

\begin{enumerate}[label=\arabic*)]

%% Es 1
\item Write an unique command (pipeline) for:
\begin{enumerate}[label=\alph*)]
\item copying the content in dir1 to dir2;
\item showing number of files contained in directories recursively (use \codeinline{ls -R} and \codeinline{find});
\item showing home's directory files which star with 3 chars and a number.
\end{enumerate}

\textbf{Solution.}
\begin{enumerate}[label=\alph*)]
\item {\ttfamily cp -r dir1/* dir2}, copies recursively the content of \codeinline{dir1} to \codeinline{dir2}
\item {\ttfamily ls -R | find | wc -l}, shows the number of lines produced by find command.

The first command executed is \codeinline{ls -R} which outputs 
	\begin{lstlisting}
	foo@vtest:~/Documents/es/studenti$ ls -R
	.:
	dir1  dir2  f1.txt  f2.txt

	./dir1:
	f1.txt  f2.txt

	./dir2:
	f1.txt  f2.txt
	\end{lstlisting}
then \codeinline{find} that outputs
	\begin{lstlisting}
	foo@vtest:~/Documents/es/studenti$ ls -R | find
	.
	./f2.txt
	./f1.txt
	./dir1
	./dir1/f2.txt
	./dir1/f1.txt
	./dir2
	./dir2/f2.txt
	./dir2/f1.txt
	\end{lstlisting}
and finally \codeinline {wc -l} which counts number of lines (i.e. numer of files and directory)
	\begin{lstlisting}
	foo@vtest:~/Documents/es/studenti$ ls -R | find | wc -l
	9
	\end{lstlisting}
\item {\ttfamily echo ~/[a-z][a-z][a-z][0-9]*} 
\end{enumerate}  

%% Es 2
\item What are the difference between:
\begin{enumerate}[label=\alph*)]
\item \codeinline{ls};
\item \codeinline{ls | cat};
\item \codeinline{ls | more}.
\end{enumerate}
\textbf{Solution.}
\begin{enumerate}[label=\alph*)]
\item \codeinline{ls} list information about files in a directory;
\item \codeinline{ls | cat} takes the output of \codeinline{ls} and prints it on stdout;
\item \codeinline{ls | more} takes the output of \codeinline{ls} and prints with \codeinline{more} format, providing filters and interactivity.
\end{enumerate}

%% Es 3
\item What are the effects of following commands?
\begin{enumerate}[label=\alph*)]
\item \codeinline{uniq < file};
\item \codeinline{who | wc -l};
\item \codeinline{ps -e | wc -l}.
\end{enumerate}
\textbf{Solution.}
\begin{enumerate}[label=\alph*)]
\item \codeinline{uniq < file}, compares adjacent lines and merge to the first occurrence matching lines in $file$q;
\item \codeinline{who | wc -l}, prints the number of lines from $input$. In this case input is \codeinline{who} which show logged in users, so \codeinline{who | wc -l} shows number of logged users;
\item \codeinline{ps -e | wc -l}, like the previos but here \codeinline{ps -e | wc -l} shows number of processes running in the system.
\end{enumerate}

%% Es 4
\item Override \codeinline{rm} command so that no delete confirmation is asked.

\textbf{Solution.} 
\begin{lstlisting}
foo@vtest:~$ alias rm="rm --interactive=never"
foo@vtest:~$ alias rmi="rm -i"
\end{lstlisting}


%% Es 5
\item Write a pipeline which outputs number of executing process.

\textbf{Solution.} 
\begin{lstlisting}
foo@vtest:~/Documents$ ps -e | wc -l
\end{lstlisting}


%% Es 6
\item Save to a file last command with \codeinline{ls}.

\textbf{Solution.} 
\begin{lstlisting}
foo@vtest:~/Documents$ !ls > t.txt
\end{lstlisting}


%% Es 7
\item Write a command which shows number of commands saved in history.

\textbf{Solution.} 
\begin{lstlisting}
foo@vtest:~/Documents$ history | wc -l
\end{lstlisting}


%% Es 8
\item Write a command which shows top 15 commands saved in history.

\textbf{Solution.} 
\begin{lstlisting}
foo@vtest:~/Documents$ history | tail -n 15
\end{lstlisting}


%% Es 9
\item What are unix commands which start with $lo$?.

\textbf{Solution.} 
\begin{lstlisting}
foo@vtest:~$ lo[TAB]
loadkeys        locate          logout          lorder
loadunimap      lodraw          logresolve      losetup
local           loffice         logrotate       loweb
localc          lofromtemplate  logsave         lowntfs-3g
locale          logger          loimpress       lowriter
localectl       login           lollipop        
localedef       loginctl        lomath          
locale-gen      logname         look        
\end{lstlisting}


%% Es 10
\item Create at least 2 ways to display home dir files starting with $al$?.

\textbf{Solution.} 
\begin{lstlisting}
foo@vtest:~$ ?????
\end{lstlisting}


%% Es 11
\item What is the result of following commands?.
\begin{enumerate}[label=\alph*)]
\item \codeinline{ls -R || (echo file non accessibili > tmp)}
\item \codeinline{(who | grep rossi) && cd ~rossi}
\item \codeinline{(cd / ; pwd ; ls | wc -l )}
\end{enumerate}

\textbf{Solution.} 
\begin{enumerate}[label=\alph*)]
\item \codeinline{(echo file non accessibili > tmp)} will be executed only if \codeinline{ls -R} fails, if everything it's fine \codeinline{ls -R} will outputted on screen;
\item if \codeinline{(who | grep rossi)} is successful (rossi is logged in), change directory to \codeinline{~rossi}
\item \codeinline{(cd / ; pwd ; ls | wc -l )} moves to root directoy, show on screen \textit{present working directory} then shows number of files and directories in /.
\end{enumerate}

\end{enumerate}



% ----------------------------------------------------------------------------
%%%%%%%%%%%%%%%%%%%%%%%%%%%%%%%%%%%%%%%%%%%%%%%%
%											   %
% 			More on Bash, Filters   		   %
%											   %
%%%%%%%%%%%%%%%%%%%%%%%%%%%%%%%%%%%%%%%%%%%%%%%%

\chapter{More on Bash, Filters}

\begin{chapquote}{08/05/2018}
	Laboratory: Lesson 3
\end{chapquote}

\section{Exercises}

\begin{enumerate}[label=\arabic*)]

%% Es 1
\item Execute \codeinline{(xterm&) ; (nice xterm&) ; (nice -n 19 xterm&); ps -el}

\textbf{Solution.} 
\begin{lstlisting}
foo@vtest:~$ (xterm&) ; (nice xterm&) ; (nice -n 19 xterm&); ps -el
0 R  1000 13190  1852  0  80   0 - 20376 -      pts/1    00:00:00 xterm
0 S  1000 13192  1852  0  90  10 - 20376 poll_s pts/1    00:00:00 xterm
0 R  1000 13194  1852  0  99  19 - 16758 -      pts/1    00:00:00 xterm
\end{lstlisting}

%% Es 2
\item What is the effect of \codeinline{sort file > file}?

\textbf{Solution.} The output of sort command is redirected and writed on file.

%% Es 3
\item What is the effect of \codeinline{tr str1 str2} if $|str1| \neq |str2|$ ?

\textbf{Solution.} $tr$ translates a text by $str1[i]$ occurence with $str2[i]$. If $str2<str1$ the the last charater of $str2$ is repeated as necessary. If $str2\geq str1$ normal transaltion of $str1$ happens.

%% Es 4
\item Replace alphanumerics characters with a <tab>.

\textbf{Solution.} \codeinline{cat b.txt | tr -s A-Za-z0-9 \\t}

%% Es 5
\item Write a pipeline that discover repeated lines in a file.

\textbf{Solution.} \codeinline{cat b.txt | sort | uniq --count}

% Es 6 
\item Write a command to discover users that has at least a process running on the system and output them once.

\textbf{Solution.}
\begin{lstlisting}
foo@vtest:~/temp/t$ ps -fe | awk ' { if ($1!="UID") { print $1 } } ' | sort | uniq
avahi
colord
foo
gdm
kernoops
message+
mysql
nvidia-+
postgres
root
rtkit
syslog
systemd+
whoopsie
www-data
\end{lstlisting}

\begin{lstlisting}
foo@vtest:~/temp/t$ ps -f | awk ' { if ($1!="UID") { print $1 } } ' | sort | uniq
foo
\end{lstlisting}

\end{enumerate}



% ----------------------------------------------------------------------------
%%%%%%%%%%%%%%%%%%%%%%%%%%%%%%%%%%%%%%%%%%%%%%%%
%											   %
% 				    Scripts         		   %
%											   %
%%%%%%%%%%%%%%%%%%%%%%%%%%%%%%%%%%%%%%%%%%%%%%%%

\chapter{Scripts}

\begin{chapquote}{22/05/2018}
	Laboratory: Lesson 4
\end{chapquote}

\section{Bash' scripts}
Shell' scripts are intepreted by shell and they are writed with atomics commands.

A script can be executed if it has $x$ permission, and the firt line must be \codeinline{#!/bin/bash}. Scripts can be executed by calling \codeinline{./script_name}.

\section{Arrays}
Assignment: \codeinline{a=(0 1 2 3 4)}, read: 


\section{Exercises}

\subsection{Draw Triangle}
Create a command named $drawtriangle$ which prints a triangle with give height.

\noindent\textbf{Solution.}
\begin{lstlisting}
#!/bin/bash

# display the usage of drawitriangle and exit with code 1
usage() {
    echo "usage: drawtriangle <height>"
    echo ""
    echo " <height>: the height of the triangle."
    echo "   must be a number"
    echo "   must be between 3 and 15"
	exit 1
}



height=$1


# test if height is given
if [ $# -ne 1 ]
then usage; fi;

# test if height is a number
re='^[0-9]+$'
if ! [[ $height =~ $re ]] ; then
   usage
fi

# test if 3 < height < 15
if test $height -lt 3 -o $height -gt 15
then usage; fi;


# drawing triangle
for i in `seq 1 $[$height-1]` # iterate on height
do
    echo -n "|"
    for j in `seq 1 $[$i-1]` # iterate on spaces between left and right
    do
	 echo -n " "
    done
    echo "\\" # draw right side
done
for i in `seq 1 $[$height+1]` # draw the base
do
    echo -n "-"
done
echo ""
\end{lstlisting}

The output:

\begin{lstlisting}
foo@vtest:~$ ./drawtriagnle.sh 100
usage: drawtriangle <height>

 <height>: the height of the triangle.
   must be a number
   must be between 3 and 15
\end{lstlisting}

\begin{lstlisting}
foo@vtest:~$ ./drawtriagnle.sh 5
|\
| \
|  \
|   \
------
\end{lstlisting}

\subsection{Clear Core}
Create a command that delete recursively all files named $core$ starting from $dir$.

\noindent\textbf{Solution.}
\begin{lstlisting}
#!/bin/bash

# display the usage of clearcore and exit with code 1
usage() {
    echo "usage: clearcore <dirpath>"
    echo ""
    echo "Deletes all file named 'core' recursively from dirpath"
    echo ""
    echo "ARGS:"
    echo " <dirpath>: directory from which deleting core files."
    exit 1
}


dirpath=$1


# test if dirpath is given
if [ $# -ne 1 ]
then usage; fi;

# test if dirpath exist and if dirpath is a directory
if test ! -e $dirpath -o ! -d $dirpath
then usage; fi;


echo "Following file will be deleted:"
echo ""
for i in `find $dirpath -name core -type f`
do
    echo $i
done
echo ""
echo "are you sure?"
select continue in "yes" "no"
do
    for i in `find $dirpath -name core -type f`
    do
		rm $i
    done
    echo ""
    echo "Deleted"
    break
done
\end{lstlisting}

The output if non valid input

\begin{lstlisting}
foo@vtest:~$ ./clearcore.sh
usage: clearcore <dirpath>

Deletes all file named 'core' recursively from dirpath

ARGS:
 <dirpath>: directory from which deleting core files.
\end{lstlisting}

and if valid

\begin{lstlisting}
foo@vtest:~$ ./clearcore.sh dir
Following file will be deleted:

dir/core
dir/d1/core

are you sure?
1) yes
2) no
#? yes

Deleted
\end{lstlisting}

\subsection{Process Utility}
Create a user-friendly handle of process which allow to show pid and command line that has created process, giving a user, show all system processes orderer by pid, provide $kill -9$ on a pid.

\end{document}
